\usepackage{tikz}
\usepackage{pgfplots}
\usepackage{pgfplotstable, filecontents, booktabs}
\pgfplotsset{compat=1.9}

\usetikzlibrary{pgfplots.groupplots}
\usepgfplotslibrary{fillbetween}
\usetikzlibrary{calc,fit,matrix,arrows,automata,positioning,shapes}
\usetikzlibrary{arrows.meta}

\pgfplotsset{select coords between index/.style 2 args={
    x filter/.code={
        \ifnum\coordindex<#1\def\pgfmathresult{}\fi
        \ifnum\coordindex>#2\def\pgfmathresult{}\fi
    }
}}

\tikzset{
 invisible/.style={opacity=0},
 visible on/.style={alt={#1{}{invisible}}},
 alt/.code args={<#1>#2#3}{%
   \alt<#1>{\pgfkeysalso{#2}}{\pgfkeysalso{#3}}
 },
}

% Annotation of triangle with slope
\newcommand{\logLogSlopeTriangle}[5]
{
 % #1. Relative offset in x direction.
 % #2. Width in x direction, so xA-xB.
 % #3. Relative offset in y direction.
 % #4. Slope d(y)/d(log10(x)).
 % #5. Plot options.

 \pgfplotsextra
 {
  \pgfkeysgetvalue{/pgfplots/xmin}{\xmin}
  \pgfkeysgetvalue{/pgfplots/xmax}{\xmax}
  \pgfkeysgetvalue{/pgfplots/ymin}{\ymin}
  \pgfkeysgetvalue{/pgfplots/ymax}{\ymax}

  % Calculate auxilliary quantities, in relative sense.
  \pgfmathsetmacro{\xArel}{#1}
  \pgfmathsetmacro{\yArel}{#3}
  \pgfmathsetmacro{\xBrel}{#1-#2}
  \pgfmathsetmacro{\yBrel}{\yArel}
  \pgfmathsetmacro{\xCrel}{\xArel}
  %\pgfmathsetmacro{\yCrel}{ln(\yC/exp(\ymin))/ln(exp(\ymax)/exp(\ymin))}
  % REPLACE THIS EXPRESSION WITH AN EXPRESSION INDEPENDENT OF \yC TO PREVENT THE 'DIMENSION TOO LARGE' ERROR.

  \pgfmathsetmacro{\lnxB}{\xmin*(1-(#1-#2))+\xmax*(#1-#2)} % in [xmin,xmax].
  \pgfmathsetmacro{\lnxA}{\xmin*(1-#1)+\xmax*#1} % in [xmin,xmax].
  \pgfmathsetmacro{\lnyA}{\ymin*(1-#3)+\ymax*#3} % in [ymin,ymax].
  \pgfmathsetmacro{\lnyC}{\lnyA+#4*(\lnxA-\lnxB)}
  \pgfmathsetmacro{\yCrel}{\lnyC-\ymin)/(\ymax-\ymin)} % THE IMPROVED EXPRESSION WITHOUT 'DIMENSION TOO LARGE' ERROR.

  % Define coordinates for \draw. MIND THE 'rel axis cs' as opposed to the 'axis cs'.
  \coordinate (A) at (rel axis cs:\xArel,\yArel);
  \coordinate (B) at (rel axis cs:\xBrel,\yBrel);
  \coordinate (C) at (rel axis cs:\xCrel,\yCrel);

  % Draw slope triangle.
  \draw[#5]   (A)-- node[pos=0.5,anchor=north,scale=0.8] {1}
              (B)--
              (C)-- node[pos=0.5,anchor=west,scale=0.8] {#4}
              cycle;
 }
}
